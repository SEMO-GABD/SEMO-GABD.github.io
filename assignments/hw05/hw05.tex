\PassOptionsToPackage{unicode=true}{hyperref} % options for packages loaded elsewhere
\PassOptionsToPackage{hyphens}{url}
\PassOptionsToPackage{dvipsnames,svgnames*,x11names*}{xcolor}
%
\documentclass[]{article}
\usepackage{lmodern}
\usepackage{amssymb,amsmath}
\usepackage{ifxetex,ifluatex}
\usepackage{fixltx2e} % provides \textsubscript
\ifnum 0\ifxetex 1\fi\ifluatex 1\fi=0 % if pdftex
  \usepackage[T1]{fontenc}
  \usepackage[utf8]{inputenc}
  \usepackage{textcomp} % provides euro and other symbols
\else % if luatex or xelatex
  \usepackage{unicode-math}
  \defaultfontfeatures{Ligatures=TeX,Scale=MatchLowercase}
\fi
% use upquote if available, for straight quotes in verbatim environments
\IfFileExists{upquote.sty}{\usepackage{upquote}}{}
% use microtype if available
\IfFileExists{microtype.sty}{%
\usepackage[]{microtype}
\UseMicrotypeSet[protrusion]{basicmath} % disable protrusion for tt fonts
}{}
\IfFileExists{parskip.sty}{%
\usepackage{parskip}
}{% else
\setlength{\parindent}{0pt}
\setlength{\parskip}{6pt plus 2pt minus 1pt}
}
\usepackage{xcolor}
\usepackage{hyperref}
\hypersetup{
            pdftitle={HW 04: Data Visualization},
            colorlinks=true,
            linkcolor=Maroon,
            filecolor=Maroon,
            citecolor=Blue,
            urlcolor=blue,
            breaklinks=true}
\urlstyle{same}  % don't use monospace font for urls
\usepackage[margin=1in]{geometry}
\usepackage{graphicx,grffile}
\makeatletter
\def\maxwidth{\ifdim\Gin@nat@width>\linewidth\linewidth\else\Gin@nat@width\fi}
\def\maxheight{\ifdim\Gin@nat@height>\textheight\textheight\else\Gin@nat@height\fi}
\makeatother
% Scale images if necessary, so that they will not overflow the page
% margins by default, and it is still possible to overwrite the defaults
% using explicit options in \includegraphics[width, height, ...]{}
\setkeys{Gin}{width=\maxwidth,height=\maxheight,keepaspectratio}
\setlength{\emergencystretch}{3em}  % prevent overfull lines
\providecommand{\tightlist}{%
  \setlength{\itemsep}{0pt}\setlength{\parskip}{0pt}}
\setcounter{secnumdepth}{0}
% Redefines (sub)paragraphs to behave more like sections
\ifx\paragraph\undefined\else
\let\oldparagraph\paragraph
\renewcommand{\paragraph}[1]{\oldparagraph{#1}\mbox{}}
\fi
\ifx\subparagraph\undefined\else
\let\oldsubparagraph\subparagraph
\renewcommand{\subparagraph}[1]{\oldsubparagraph{#1}\mbox{}}
\fi

% set default figure placement to htbp
\makeatletter
\def\fps@figure{htbp}
\makeatother

\usepackage{etoolbox}
\makeatletter
\providecommand{\subtitle}[1]{% add subtitle to \maketitle
  \apptocmd{\@title}{\par {\large #1 \par}}{}{}
}
\makeatother

\title{HW 04: Data Visualization}
\providecommand{\subtitle}[1]{}
\subtitle{Graphical Analysis of Biological Data}
\author{}
\date{\vspace{-2.5em}}

\begin{document}
\maketitle

By the end of this assignment, you should demonstrate an ability to

\begin{itemize}
\tightlist
\item
  produce graphs in \texttt{ggplot2}, including boxplots, scatterplots,
  and line graphs.
\end{itemize}

Click on any blue text to visit the external website.

You will make 10 plots for this assignment, using the skills you
developed in the previous homework.

\textbf{Note:} If you contact me for help or (better yet) open an issue
in the \href{https://github.com/SEMO-GABD/public_discussion}{public
discussion forum,} please include the code that is not working and also
tell me what you have tried.

\hypertarget{preparation}{%
\subsection{Preparation}\label{preparation}}

\begin{itemize}
\item
  Open your \texttt{.Rproj} project file in RStudio.
\item
  Create an \texttt{hw05} folder inside the same folder as your project
  file.
\item
  Review \href{https://r4ds.had.co.nz/data-visualisation.html}{R4ds
  Chapter 3: Data Visualisation} as necessary to make the plots
  described below.
\end{itemize}

\hypertarget{graph-some-biological-data}{%
\subsection{Graph some biological
data}\label{graph-some-biological-data}}

\begin{itemize}
\item
  Create a new R Notebook called
  \texttt{\textless{}lastname\textgreater{}\_hw05.Rmd} and save it in
  your \texttt{hw05} folder.
\item
  Copy and paste the YAML header from HW04 and replace the default
  header in your new document. Change the title as appopriate.
\item
  Load the \texttt{tidyverse} package in your first code chunk.
\end{itemize}

Develop this habit for the remaining assignments: Open your Rproj file,
download or create your new notebook as assigned, edit the YAML file,
and then insert your first code chunk where you will load any packages
needed for the assignment.

\begin{center}\rule{0.5\linewidth}{\linethickness}\end{center}

For this homework, you will use \texttt{ggplot2} to make plots from some
of the datasets that come with R and the \texttt{tidyverse} packages. I
will give you the dataset to use, and other information to use for
mapping, etc. I expect you will write and execute the code.

\begin{itemize}
\item
  The first time you use a dataset, load it with the command
  \texttt{data(dataset\ name)} in your code chunk. For example,
  \texttt{data(faithful)} loads the Old Faithful dataset. Technically,
  you do not have to do this but it is good coding practice. \textbf{I
  expect that you will do this.}
\item
  After you load the dataset, and only for the first time, enter
  \texttt{?\textless{}dataset\ name\textgreater{}} in your code chunk to
  see the format of the data. For example, \texttt{?faithful} will give
  you information about the Old Faithful dataset.

  \textbf{Note:} You should \emph{always} inspect your data visually.
  That is why I am telling you to do this step.

  You only need to do these two steps the first time you use a dataset.
\item
  Run each code chunk, and write 1-2 sentences that describes any trends
  or patterns that you observe in the plot. In other words, think like a
  scientist!
\item
  Include the \texttt{\#\#\#\#\ Plot\ \textless{}no.\textgreater{}}
  header above each plot.
\end{itemize}

\hypertarget{plot-1}{%
\paragraph{Plot 1}\label{plot-1}}

\begin{itemize}
\tightlist
\item
  Plot type: scatterplot
\item
  Dataset: trees
\item
  x-axis: height
\item
  y-axis: girth
\end{itemize}

\hypertarget{plot-2}{%
\paragraph{Plot 2}\label{plot-2}}

Apply some of your skills that you learned during Assignment 02. You
will make two vectors, then combine them into a data frame for plotting.
Review the assignments if necessary.

\begin{itemize}
\item
  Make a vector called \texttt{year} for 1821 to 1934. Rememeber how to
  use \texttt{:} to make a sequence of numbers?
\item
  Look at the \texttt{class()} of the \texttt{lynx} dataset. The
  \texttt{lynx} dataset is a ``time series'' class (\texttt{ts}). You
  can convert the time series data to a vector by using the
  \texttt{as.vector()} function. Just put the dataset name inside the
  parentheses. Assign this to the variable \texttt{pelts}.
\item
  Make a dataframe called \texttt{lynx\_pelts} from these two vectors.
\item
  Plot type: linegraph
\item
  Dataset: lynx\_pelts (see above)
\item
  x-axis: year
\item
  y-axis: pelts
\item
  Make the line color maroon. Maroon is one of the default R colors.
\end{itemize}

\hypertarget{plot-3}{%
\paragraph{Plot 3}\label{plot-3}}

\begin{itemize}
\tightlist
\item
  Plot type: scatterplot
\item
  Dataset: iris
\item
  x-axis: petal length
\item
  y-axis: petal width
\item
  Point color and shape by species. You do not have to use fillable
  shapes.
\item
  Point size of 2
\item
  Add a \texttt{labs} layer to change the x- and y-axis labels so that
  they do not have periods in the names (i.e., \texttt{Petal\ Length},
  \texttt{Petal\ Width}).
\end{itemize}

\hypertarget{plots-4-and-5}{%
\paragraph{Plots 4 and 5}\label{plots-4-and-5}}

\begin{itemize}
\item
  This requires two code chunks, which will be nearly identical
\item
  Plot type: Violin plot (look up \texttt{geom\_violin})
\item
  Dataset: msleep
\item
  x-axis: vore
\item
  y-axis: sleep\_rem
\item
  fill = grayXX where XX is either 30 or 70.
\item
  In your description, describe in your own words what violin plots
  display (you can search the interwebs), and what is the difference
  among the two versions of gray shading. \emph{Hint:} the grays extend
  from \texttt{gray0} to \texttt{gray100}. You can learn more about
  colors in R from
  \href{https://www.nceas.ucsb.edu/~frazier/RSpatialGuides/colorPaletteCheatsheet.pdf}{this
  PDF file.}
\end{itemize}

\hypertarget{plot-7}{%
\paragraph{Plot 7}\label{plot-7}}

\begin{itemize}
\tightlist
\item
  Plot type: boxplot
\item
  Dataset: msleep
\item
  x-axis: order
\item
  y-axis: sleep\_total
\item
  use \texttt{coord\_flip()}
\end{itemize}

\hypertarget{plot-8}{%
\paragraph{Plot 8}\label{plot-8}}

\begin{itemize}
\tightlist
\item
  Plot type: boxplot with points
\item
  Dataset: msleep
\item
  x-axis: conservation
\item
  y-axis: awake
\item
  You must have a boxplot layer, a point layer, and a jitter layer.
\item
  color by conservation status.
\item
  You \emph{may} use \texttt{coord\_flip} but it is not required. Try
  both and choose the one you think looks best.
\item
  Add a \texttt{lab} layer to change both axis labels so each starts
  with an upper-case letter.
\item
  Search the internet to find out how to change the legend title to
  \texttt{Conservation}. Make that change. (Do not try to change the
  actual legend entries like ``cd'' and ``vu''). \textbf{Note:} This can
  be done a couple of different ways but
  \texttt{scale\_color\_discrete()} is one good way.
\end{itemize}

\hypertarget{plots-9-and-10.}{%
\paragraph{Plots 9 and 10.}\label{plots-9-and-10.}}

\begin{itemize}
\item
  Make two scatterplots of your choice, with the following constraints.
\item
  One should plot any one of the ``sleep'' or ``awake'' variables
  against body weight. The other should plot any one of the ``sleep'' or
  ``awake'' variables against brain weight.
\item
  In each plot, color the points by one of the nominal data categories.
  Use a different category for each plot.
\item
  Apply \texttt{facet\_wrap()} to at least one of the plots using one of
  the nominal variables. You decide whether you use 2 or 3 columns.
  \emph{Hint:} use one of the nominal variables with relatively few
  different types for wrapping. \emph{Explore:} What happens if you use
  a nominal variable like \texttt{genus}, with lots of types?
\item
  You should try a few versions of each graph until you find
  combinations that you think show some clear trends.
\item
  Describe the patterns or trends you see in each graph.
\end{itemize}

\end{document}
